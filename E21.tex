\documentclass{article}
\usepackage[utf8]{inputenc}

\title{AGT}
\author{Tim Wieschalla}
\date{November 2017}

\usepackage{natbib}
\usepackage{graphicx}

\begin{document}

\maketitle

\section{21.}

 Consider an auction in which the seller announces a reserve price of r before running the auction. With a reserve price, the item is sold to the highest bidder if the highest bid is above r; otherwise, the item is not sold. In a first-price auction with a reserve price, the winning bidder (if there is one) still pays her bid. In a second-price auction with a reserve price, the winning bidder (if there is one) pays the maximum of the second-place bid and the reserve price r.
\\
\\
• Is truthful bidding a dominant strategy on such auctions?\\
• Assuming that the seller assigns value u to the object and chooses the reserve price r in order to maximize its expected revenue in a SP auction. Show that r might be higher than u. (Hint. Provide a value for the simple case in which the item is worth u = 0 to the seller and a second-price auction with a single bidder, whose value is uniformly distributed on [0, 1] is run.)




\section{Solution}
1. first-price auction with a reserve price:\\
recall def first-price auction: \\
highest bid wins
\\\\
no, because counter-example:\\
one bidder b1\\
r=10 \\
truthfull bid(real value for b1) v(b1)= 15\\
$\rightarrow$ b1 loses/negative revenue of 5\\
$\rightarrow$ b1 will bid 10, also bc he knows r\\
$\rightarrow$ truthfulness is not dominant\\
% (-> unthruthful bidding is dominant?)
\\\\
2. second-price auction with a reserve price

% ---
% v1 WRONG!
% recall def second-price auction: 
% second higest bid wins

% no, bc. example:
% two bidder
% % r = 10
% % v(b1) = 15
% % v(b2) = 12
% % b1 wins, pays 12
% % -> truthful dominant

% % r = 10
% % v(b1) = 9
% % v(b2) = 8 
% % -> b1 wins, pays 10
% % -> but wouldnt bid 9, bc he know r xD

% r = 10
% v(b1) = 9
% v(b2) = 8 
% -> b1 wins, pays 10
% -> bidding 0 would have been better
% %-> b1,b2 will bid 0, also bc they knows r <- no they won't because they want to get it (but it doesn't matter for the exercise)
% -> truthfulness is not dominant

% % TODO ask julian if this was the right example :/

% ---
% v2:
% TODO check slide
% idea 1: changes some things
idea: induction over amount of player:\\
bX = bids\\
vX = real values for the player\\
\\\\
one bidder:\\
n=1\\
...\\
induction:\\
n+1:\\
2 cases for untruthfull:\\
-v < b: \\
...\\
-v > b: \\
...\\



% r = 10
% v(b1) = 12
% v(b2) = 8
% -> b1 wins, pays 10
% -> truthful dominant



3. show that r might be higher than u\\
"assumptions":\\
u = 0\\
single bidder\\
V value of bidder\\
V is uniformly distributed on [0, 1]\\
only sold if b >= r\\
\\\\

% ---
% solution v1:
% % TODO explain what this means and what follows
% % das bedeutet, dass der Wert für den Bieter gleichverteilt zwischen 0 und 1 ist.
% % Zum Beispiel will er den Gegenstand wieder verkaufen und schätzt ein, dass er einen Wert zwischen 0 und 1 erreichen wird, weiß aben noch nicht welchen.
% % Nur auf Grundlage dieser Einschätzung kann er sein Gebot machen. Er weiß also nicht wie viel es tatsächlich wert sein wird zum Zeitpunkt seines Gebots.
% % Und er will E[v] - b maximeiren wo v sein wert ist und b sein gebot

% E[V] in [0,1] = (0+1)/2 = 0,5
% bc. Uniform distribution (continuous)
% recall E(X) = 1/2(a*b)

% 0 < r, bc r of 0 is giving it for free

% E[V]-r is utility bc trivial

% seller gains r

% cases:
% r > 0,5:  utility is negative, bc 0,5(this is E[V]) -r is negative
% r = 0,5:  utility is zero, so buyer is free to decide, and the seller cant know if the buyer will buy, so seller will evade this case
% r < 0,5: utility is positive, bc 0,5(this is E[V]) -r is postive

% but r < 0,5 , bc if its 0,5 the bidder can do whatever he wants. so not good for seller, so
% -> r < 0,5
% -> 0 < r < 0,5

% conclusion: the bidder will only bid if r is 0 < r < 0,5, bc postive utility.
% BUT: the bid itself doesnt matter, bc SP-auction and he is the only bidder

% % ------------------



% %( its the intergral of f(x)=1 in the bounds [0, 1])

% value * probablity
% % explain why:
% % value rises in the bounds from 0->1
% % -> f(x)=x
% % probability of ?


% f(r)=r * (r-1)= r^2 - r
% r_{higher tha u}=f(x_max)=0,5
% % TODO explain functions+graphs

% -> 0,5-\epsilon > 0 = r > u






% % - truthful bidding a dominant: yes, an initially reserved price doesnt change this and truthful bidding a dominant strategy on first and 2nd price auctions
% %TODO formal ausdrücken
% \\
% % - ?! mir fehlt der ansatz...
% % also mit u = 0 anfangen und dann zeigen, dass r > u sein kann.


% ---
% solution v2:

At first we have to consider 2 cases:\\
if for bid b and reserved price r:\\
1. $b<r$ then the item is not sold, payment is 0\\
2. $b>r$ then seller receives r\\
\\\\
Expected revenue:\\
$E(r) = 0 * r + r * (1-r)$\\
0 and r are payments\\
r and 1-r are probabilities:\\
$P[v<r] = r$\\
$P[v>r] = 1-r$\\
% TODO Reason?\\
% bc.: \\
% ...\\

this means:\\
E(r)= r - r^2\\
E'(r)= 1 - 2*r\\
so the maximum is at r = 1/2\\

% TODO write whole sentences

so we can conclude:\\

% TODO check if still right
\\\\
cases:\\
$r > 0,5$:  utility is negative, bc 0,5(this is E[V]?) -r is negative\\
$r = 0,5$:  utility is zero, so buyer is free to decide, and the seller cant know if\\ the buyer will buy, so seller will evade this case\\
$r < 0,5$: utility is positive, bc 0,5(this is E[V]?) -r is postive\\
\\\\
but r < 0,5 , bc if its 0,5 the bidder can do whatever he wants. so not good for seller, so\\
$\rightarrow$ r < 0,5\\
$\rightarrow$ 0 < r < 0,5\\
\\\\
conclusion: the bidder will only bid if r is $0 < r < 0,5$, bc postive utility.
BUT: the bid itself doesnt matter, bc SP-auction and he is the only bidder

\end{document}
